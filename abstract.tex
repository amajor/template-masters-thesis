\addcontentsline{toc}{section}{Abstract}
\section*{Abstract} \label{sectionAbstract}

% State the problem.
% Say why this problem is interesting.
% Say what my solution achieves.
% Say what follows from my solution.

All the pages have been formatted in the accepted font and margin alignment. This is a simple MS thesis template that can be used for directly typing in your content. However, if you paste your text into the document, do so with caution as pasting could produce varying results. When directly typing into the title page and signature page, the appropriate information should be filled in the required fonts.  If one chooses to include a copyright notice, it should appear before the signature page and after the title page (page ii). This can be achieved by clicking Insert > break > page break >ok.  Additionally, the page number should not appear on the copyright notice page. This can be achieved by clicking Insert > page numbers > format > start numbering at. I have used this thesis template to answer typical questions that grad students need addressed before they begin writing their theses. When writing an abstract, bare in mind an abstract is a short descriptive summary of your thesis. The number of words accepted might vary e.g. 200-250 words. An MS thesis abstract need not exceed two pages. \textbf{Abstracts are typically written last although they are the most important part of the thesis. They should have a little bit of everything: the background, the scope of your project, the purpose, findings and conclusions. An abstract is neither paragraphed nor cited. It should not be written as a literature review or a discussion of results. In a simplistic manner, your abstract, in a few words, should answer the questions: why should we care about your research; how did you get your results; what did you learn, find, create, invent; and finally what do your results imply?}