\documentclass{beamer}

% =============================================================================
% =  CUSTOM INFORMATION (stored as variables, used in template)
% =============================================================================

% The title of the paper.
\newcommand{\paperTitle}{My Thesis Title}

% The name of the author.
\newcommand{\paperAuthor}{Alison Major}

% The name of the author.
\newcommand{\authorInstitute}{Lewis University}

\usetheme{Madrid}
\usecolortheme{beaver}

% Set graphics location
\graphicspath{ {Images/} }



%Information to be included in the title page:
\title{\paperTitle}
\author{\paperAuthor}
\institute{\authorInstitute}
\date{\the\year{}}

% \logo{
%   \includegraphics[width=0.2\textwidth]{UniversityLogo}
% }

\begin{document}

\frame{\titlepage}

% 1 Introduction
% Chapter one defines the overall importance of the problem areas and provides an introduction into what you did.
\begin{frame}
  \frametitle{Introduction}
  Chapter one defines the overall importance of the problem areas and provides an introduction into what you did.
\end{frame}

% 2 Background and Literature Review
% Chapter two is why you did it in the context of what was previously known.
\begin{frame}
  \frametitle{Background and Literature Review}
  Chapter two is why you did it in the context of what was previously known.
\end{frame}

% 3 Methodology
% Chapter three is how you did it.
\begin{frame}
  \frametitle{Methodology}
  Chapter three is how you did it.
\end{frame}

% 4 Results
% Chapter four is what you found.
\begin{frame}
  \frametitle{Results}
  Chapter four is what you found.
\end{frame}

% 5 Conclusions and Recommendations
% Chapter five is what it all means.
\begin{frame}
  \frametitle{Conclusions and Recommendations}
  Chapter five is what it all means.
\end{frame}

\end{document}